\begin{longtable}{|>{\raggedright\arraybackslash}p{0.12\textwidth}|>{\raggedright\arraybackslash}p{0.46\textwidth}|>{\raggedright\arraybackslash}p{0.20\textwidth}|>{\raggedright\arraybackslash}p{0.16\textwidth}|}
    \caption{List of proteins with corresponding sequences, descriptions, and sources.} \\
    \hline
    \textbf{Part} & \textbf{Sequence} & \textbf{Description} & \textbf{Source/Part ID} \\
    \hline
    \endfirsthead

    \hline
    \textbf{Part} & \textbf{Sequence} & \textbf{Description} & \textbf{Source/Part ID} \\
    \hline
    \endhead

    \hline
    \multicolumn{4}{r}{\textit{Continued on next page}}                                    \\
    \endfoot

    \hline
    \endlastfoot

    RR1234L-VP16 &
    {\small\ttfamily\raggedright
            \seqsplit{ATGAAGGGAGGAGGACTCGAGATTAGAGCTGCTTTCCTCAGAAGAAGAAACACAGCTCTCAGAACAAGAGTTGCTGAGCTCAGACAAAGAGTTCAAAGACTCAGAAACATTGTTTCTCAATACGAGACAAGATACGGACCACTCAGTACAGCACCTCCAACCGATGTAAGCCTTGGCGATGAGCTCCATTTGGATGGAGAAGATGTTGCAATGGCTCACGCAGATGCCCTTGATGATTTTGACCTCGATATGTTGGGAGATGGCGATTCGCCTGGTCCAGGTTTCACTCCTCACGACTCTGCTCCTTACGGCGCACTTGATACTGCAGATTTCGAGTTCGAGCAAATGTTCACTGATGCCCTCGGCATTGATGAATACGGTGGTTAG}
            \par}
                      &
    Codon optimised RR1234L is the basic half of the split coiled-coil dimerization motif, needed for VP16-GAL4 fusion. VP16 is a widely used strong activation domain that efficiently recruits eukaryotic transcriptional machinery and has been shown to function in plants. Codon optimised using NovoPro.
                      &
    RR1234L \citep{Moll2001}, VP16 (BBa\_K3242005) \\
    \hline

    TetR-Linker-SRDX-PEST &
    {\small\ttfamily\raggedright
            \seqsplit{ATGGCTAGACTCAACAGAGAGTCTGTTATTGATGCTGCTCTCGAGCTCCTCAACGAGACAGGAATTGATGGACTCACAACAAGAAAGCTCGCTCAAAAGCTCGGAATTGAGCAACCAACACTCTACTGGCACGTTAAGAACAAGAGAGCTCTCCTCGATGCTCTCGCTGTTGAGATTCTCGCTAGACACCACGATTACTCTCTCCCAGCTGCTGGAGAGTCTTGGCAATCTTTCCTCAGAAACAACGCTATGTCTTTCAGAAGAGCTCTCCTCAGATACAGAGATGGAGCTAAGGTTCACCTCGGAACAAGACCAGATGAGAAGCAATACGATACAGTTGAGACACAACTCAGATTCATGACAGAGAACGGATTCTCTCTCAGAGATGGACTCTACGCTATTTCTGCTGTTTCTCACTTCACACTCGGAGCTGTTCTCGAGCAACAAGAGCACACAGCTGCTCTCACAGATAGACCAGCTGCTCCAGATGAGAACCTCCCACCACTCCTCAGAGAGGCTCTCCAAATTATGGATTCTGATGATGGAGAGCAAGCTTTCCTCCACGGACTCGAGTCTCTCATTAGAGGATTCGAGGTTCAACTCACAGCTCTCCTCCAAATTGTTGGAGGAGATAAGCTCATTATTCCATTCTGCGGATCTGGATTGGACCTTGATCTTGAATTGAGACTTGGTTTTGCATCGGGGTCCGGCAGCCACGGTTTTCCACCTGAGGTCGAGGAACAGGCGGCAGGAACCCTGCCCATGTCCTGCGCTCAGGAGTCTGGTATGGACAGACATCCCGCTGCATGTGCAAGCGCCAGAATTAACGTGTAG}
            \par}
                      &
    TetR is a transgenic bacterial repressor that binds and inhibits the TetO operator. It is widely used in synthetic biology as a NOT gate and has been demonstrated to function in plant systems. Codon optimised. The Gly-Ser-Gly (GSG) linker has been demonstrated to link protein domains without interfering with function or folding. Codon optimised. SRDX is a plant repression domain derived from plant transcriptional repressors to silent gene expression. PEST degradation tags are used in plant synthetic biology to accelerate protein turnover. An additional stop codon has been added.
                      &
    TetR from UniProt P0ACT4, GSG linker \citep{Zhang2022}, SRDX and PEST \citep{Ferreira2024,SarrionPerdigones2013}, Genbank JQ437371.1 \\
    \hline

    GAL4-EE1234L &
    {\small\ttfamily\raggedright
            \seqsplit{ATGAAGTTGCTCTCTAGCATAGAACAAGCTTGCGATATCTGTCGACTCAAGAAGTTGAAGTGTTCCAAGAAAAAGCCTAAATGCGCAAAGTGCCTTAAGAATAATTGGGAATGCAGGTACTCACCAAAGACTAAAAGAAGCCCATTGACACGAGCTCATTTGACTGAGGTCGAAAGTCGTTTGGAGAGATTAGAACAGCTCTTTTTGTTGATCTTCCCTCGTGAAGATCTTGACATGATCTTGAAGATGGACTCTTTACAAGACATCAAAGCACTGCTCACAGGTCTGTTTGTCCAGGACAACGTTAACAAGGACGCAGTGACTGACAGACTTGCTTCAGTCGAAACAGATATGCCATTGACTTTGCGTCAGCATAGGATATCCGCGACGTCTTCTTCTGAGGAAAGTAGCAATAAAGGGCAACGACAGTTGACTGTTCTCGAGATTGAGGCTGCTTTCCTCGAGCAAGAGAACACAGCTCTCGAGACAGAGGTTGCTGAGCTCGAGCAAGAGGTTCAAAGACTCGAGAACATTGTTTCTCAATACGAGACAAGATACGGACCACTCGGAGGAGGAAAGTAG}
            \par}
                      &
    GAL4 is widely used in synthetic biology alongside VP16 as a split transcription factor and has been demonstrated to work in plants. Codon optimised using NovoPro. EE1234L constitutes the acidic half of the split coiled-coil dimerization motif, needed for GAL4/VP16 association. An additional stop codon has been added. Codon optimised.
                      &
    GAL4 from BBa\_K3242004, EE1234L \citep{Moll2001} \\
    \hline

    GAL4/VP16 fusion &
    {\small\ttfamily\raggedright
            \seqsplit{ATGAAGCTCCTGTCCTCCATCGAGCAGGCCTGCGACATCTGCCGCCTCAAGAAGCTCAAGTGCTCCAAGAAGAAGCCGAAGTGCGCCAAGTGTCTGAAGAACAACTGGGAGTGTCGCTACTCTCCCAAAACCAAGCGCTCCCCGCTGACCCGCGCCCACCTCACCGAAGTGGAGTCCCGCCTGGAGCGCCTGGAGCAGCTCTTCCTCCTGATCTTCCCTCGAGAGGACCTCGACATGATCCTGAAAATGGACTCCCTCCAGGACATCAAAGCCCTGCTCACCGGCCTCTTCGTCCAGGACAACGTGAACAAAGACGCCGTCACCGACCGCCTGGCCTCCGTGGAGACCGACATGCCCCTCACCCTGCGCCAGCACCGCATCAGCGCGACCTCCTCCTCGGAGGAGAGCAGCAACAAGGGCCAGCGCCAGTTGACCGTCTCGACGGCCCCCCCGACCGACGTCAGCCTGGGGGACGAGCTCCACTTAGACGGCGAGGACGTGGCGATGGCGCATGCCGACGCGCTAGACGATTTCGATCTGGACATGTTGGGGGACGGGGATTCCCCGGGGCCGGGATTTACCCCCCACGACTCCGCCCCCTACGGCGCTCTGGATACGGCCGACTTCGAGTTTGAGCAGATGTTTACCGATGCCCTTGGAATTGACGAGTACGGTGGGTAG}
            \par}
                      &
    A hybrid transcription factor used by UGA iGEM team in 2019 for agrobacterium-based transformation of N. benthamiana.
                      &
    BBa\_K3242006 \\
    \hline

    LexA-PEST &
    {\small\ttfamily\raggedright
            \seqsplit{ATGAAAGCGTTAACGGCCAGGCAACAAGAGGTGTTTGATCTCATCCGTGATCACATCAGCCAGACAGGTATGCCGCCGACGCGTGCGGAAATCGCGCAGCGTTTGGGGTTCCGTTCCCCAAACGCGGCTGAAGAACATCTGAAGGCGCTGGCACGCAAAGGCGTTATTGAAATTGTTTCCGGCGCATCACGCGGGATTCGTCTGTTGCAGGAAGAGGAAGAAGGGTTGCCGCTGGTAGGTCGTGTGGCTGCCTCGGGGTCCGGCAGCCACGGTTTTCCACCTGAGGTCGAGGAACAGGCGGCAGGAACCCTGCCCATGTCCTGCGCTCAGGAGTCTGGTATGGACAGACATCCCGCTGCATGTGCAAGCGCCAGAATTAACGTGTAA}
            \par}
                      &
    The Addgene-derived LexA module encodes a bacterial DNA-binding protein (LexA) that recognizes and binds LexO operator sites in plant systems. This part is derived from the GoldenBraid plant synthetic biology framework. PEST tag included. An additional stop codon has been added.
                      &
    Addgene \#68184 \citep{SarrionPerdigones2013}, PEST from Genbank JQ437371.1 \\
    \hline

    ADH1-F2A-PDC-PEST &
    {\small\ttfamily\raggedright
            \seqsplit{ATGTCTAATACTGCTGGTCAGGTCATACGCTGCAGAGCTGCTGTAGCTTGGGAAGCAGGGAAGCCACTGGTGATTGAAGAAGTTGAGGTGGCACCACCACAAGCAAATGAAGTTCGCATAAAGATCCTTTTTACATCTTTGTGCCACACTGATGTCTACTTCTGGGAAGCCAAGGGACAAAACCCTTTATTTCCTAGAATTTATGGTCATGAGGCAGGAGGGATTGTGGAGAGTGTTGGTGAGGGCGTGACGGATCTGAAAGCCGGCGATCATGTCCTGCCGGTGTTCACAGGGGAATGCAAGGACTGCGCTCACTGCAAATCAGAAGAGAGCAACATGTGTGACCTCCTCAGGATAAACACTGACAGGGGAGTGATGCTCAGTGATGGAAAATCAAGATTTTCAATCAAAGGCAAGCCTATCTACCATTTTGTTGGGACTTCCACCTTCAGCGAGTACACTGTTGTTCACGTTGGCTGCCTTGCCAAGATCAATCCCTCGGCGCCTCTAGACAAAGTCTGTCTCCTCAGTTGTGGAATCTCCACAGGTCTCGGAGCTACTCTAAATGTTGCAAAACCAAAAAAGGGATCAACCGTGGCTGTTTTCGGATTGGGAGCTGTAGGCCTTGCAGCTGCTGAAGGAGCCAGGTTGTCTGGCGCTTCAAGAATTATCGGTGTTGATTTGCATTCGGACAGATTTGAAGAAGCAAAAAAGTTTGGCGTGACAGAATTCGTGAACCCAAAAGCGCACGAAAAACCAGTTCAAGAGGTGATTGCTGAGTTGACGAATCGAGGAGTGGACAGAAGCATTGAATGTACAGGAAGCACTGAAGCCATGATATCTGCATTTGAATGTGTCCATGATGGTTGGGGTGTTGCTGTTCTTGTGGGAGTACCACACAAAGATGCCGTCTTCAAGACGCATCCGGTTAACTTTCTGAATGAGAGGACTCTCAAGGGTACATTCTTCGGAAACTACAAGACTCGAACGGACATTCCCTCTGTCGTGGAGAAGTACATGAACAAGGAACTGGAGCTAGAGAAATTCATCACCCACAAAGTCCCGTTCTCAGAAATCAACAAGGCATTTGAGTACATGCTTAAAGGGGAAGGTCTTCGTTGCATAATCCGCATGGAGGAATGACAACTCCTCAACTTCGATCTCCTCAAGCTCGCTGGAGATGTTGAGTCTAACCCAGGACCAATGGACACCAAAATTGGTTCGCTTGACGTCTGCAAGCCTACGTGCACCGGCGTCGGCAGCCTACCGAACGGCGCCGCTTTAGCAATCCAAAGCTCTGCCCCCTCCCTCATCAACTCCTCTGACGCCACTCTGGGTGGCCACATCGCCCGCCGACTTGTCCAAATCGGCGTCACGGACGTGTTCACTGTCCCAGGTGACTTTAACTTAACCCTCCTAGACCACCTCATTGCCGAGCCTGGGCTCACCAACATCGGCTGCTGCAACGAACTCAATGCCGGGTACGCTGCTGACGGCTACGCTCGGTCGCGGGGAGTCGGGGCGTGTGTTGTTACTTTCACTGTGGGTGGGCTCAGTGTTCTCAATGCTATCGCGGGAGCTTACAGTGAGAGTCTGCCATTGATTTGTATAGTTGGAGGACCCAACTCGAATGATTACGGGACGCACAGGATTCTTCACCACACTATTGGGTTACCGGATTTTAGCCAAGAGTTGACATGCTTCCAGACCGTCACTTGCTATCAGGCTGTGGTAAATAATCTGGAAGATGCTCATGAAATGATTGATACCGCAATTTCAACCGCCTTGAAAGAAAGCAAGCCTGTTTATATCAGCATAAGCTGCAACTTGGCTGGAATTGCTCATCCAACTTTTAGCCTGGATCCTGTTCCCTTCTCATTGTCTCCAAGATTGAGTAATCATTTGGGCTTAGAGGCTGCCGTGGAGGCGGCTGCAGAGTTCTTTAACAAGGCAGTGAAGCCGGTTATGGTAGGGGGGCCTAAACTTCGAGTTGCACATGCTGGCGATGCCTTTGTTGAACTAGCAGATGCTAGTGGTTATGCGCTCGCTGTCATGCCATCTGCAAAGGGCCTTGTGCCAGAGCACCACCCCCATTTCATTGGAACATACTGGGGTGCTGTGAGCACTGCCTTTTGCGCCGAGATTGTGGAGTCCGCAGATGCATACTTGTTTGCTGGACCGATTTTCAATGACTACAGCTCTGTTGGATACTCTCTGCTTCTCAAGAAAGAGAAGGCAATTGTGGTGCAGCCTGATCACGTGACCATAGCAAATGGCCCTTCATTTGGTTGTGTTCTCATGAAGGATTTCCTCCGAGCTCTGGCAAAGAGGCTCAAGCACAACAAAACTGCTCATGAGAACTACAGCAGGATCTTTGTTCCCAACGGACACCCTCTAAAGTCTGCACCGAAAGAACCTTTGAGGGTTAATGTTCTGTTCCACCACATCCAGAATATGCTGTCAAGTGAAACTGCTGTGATTGCTGAGACAGGGGACTCATGGTTTAACTGCCAGAAACTGAAATTGCCGGCTGGTTGCGGGTATGAGTTCCAAATGCAGTATGGATCAATTGGTTGGTCAGTTGGAGCTACTCTTGGGTATGCTCAAGCTGTTACTGAGAAGCGTGTGATTGCTTTCATCGGCGATGGGAGTTTCCAGGTGACTGTTCAAGATGTGTCCACCATGATCCGAAATGGGCAGAAGAACATCATCTTCCTGATAAACAACGGCGGATACACAATTGAGGTGGAGATCCATGACGGACCATACAATGTGATCAAGAACTGGAACTACACTGGACTAGTTGATGCCATCCACAACGGGGAGGGCAAGTGCTGGACAACCAAGGTCCGTTGCGAAGAGGAGCTGATTGAAGCGATTGAGACTGCAACAGGGGCGAAGAAGGATAGCTTGTGCTTCATTGAGGTGATAGCCCACAAGGACGATACCAGCAAAGAGTTGCTTGAGTGGGGGTCTAGGGTTTCTGCTGCCAACAGCCGCCCACCCAGCCCTCAGTCGGGGTCCGGCAGCCACGGTTTTCCACCTGAGGTCGAGGAACAGGCGGCAGGAACCCTGCCCATGTCCTGCGCTCAGGAGTCTGGTATGGACAGACATCCCGCTGCATGTGCAAGCGCCAGAATTAACGTGTAA}
            \par}
                      &
    ADH1 protein sequence from Granny Smith from ATG to Stop codon. F2A linker for polycistronic expression. PDC sequence from M. domestica. PEST degradation tag included. An additional stop codon has been added.
                      &
    ADH1 from UniProt P48977, F2A \citep{Burn2012}, PDC from KEGG 103425939 \\
    \hline

    kanR &
    {\small\ttfamily\raggedright
            \seqsplit{ATGAGCCATATTCAACGGGAAACGTCTTGCTCGAGGCCGCGATTAAATTCCAACATGGATGCTGATTTATATGGGTATAAATGGGCTCGCGATAATGTCGGGCAATCAGGTGCGACAATCTATCGATTGTATGGGAAGCCCGATGCGCCAGAGTTGTTTCTGAAACATGGCAAAGGTAGCGTTGCCAATGATGTTACAGATGAGATGGTCAGACTAAACTGGCTGACGGAATTTATGCCTCTTCCGACCATCAAGCATTTTATCCGTACTCCTGATGATGCATGGTTACTCACCACTGCGATCCCCGGGAAAACAGCATTCCAGGTATTAGAAGAATATCCTGATTCAGGTGAAAATATTGTTGATGCGCTGGCAGTGTTCCTGCGCCGGTTGCATTCGATTCCTGTTTGTAATTGTCCTTTTAACAGCGATCGCGTATTTCGTCTTGCTCAGGCGCAATCACGAATGAATAACGGTTTGGTTGATGCGAGTGATTTTGATGACGAGCGTAATGGCTGGCCTGTTGAACAAGTCTGGAAAGAAATGCATAAGCTTTTGCCATTCTCACCGGATTCAGTCGTCACTCATGGTGATTTCTCACTTGATAACCTTATTTTTGACGAGGGGAAATTAATAGGTTGTATTGATGTTGGACGAGTCGGAATCGCAGACCGATACCAGGATCTTGCCATCCTATGGAACTGCCTCGGTGAGTTTTCTCCTTCATTACAGAAACGGCTTTTTCAAAAATATGGTATTGATAATCCTGATATGAATAAATTGCAGTTTCATTTGATGCTCGATGAGTTTTTCTAA}
            \par}
                      &
    Kanamycin resistance gene for selection of the construct during Golden braid assembly.
                      &
    BBa\_K3447004 \\
    \hline

    smrR &
    {\small\ttfamily\raggedright
            \seqsplit{ATGCGCTCACGCAACTGGTCCAGAACCTTGACCGAACGCAGCGGTGGTAACGGCGCAGTGGCGGTTTTCATGGCTTGTTATGACTGTTTTTTTGGGGTACAGTCTATGCCTCGGGCATCCAAGCAGCAAGCGCGTTACGCCGTGGGTCGATGTTTGATGTTATGGAGCAGCAACGATGTTACGCAGCAGGGCAGTCGCCCTAAAACAAAGTTAAACATCATGAGGGAAGCGGTGATCGCCGAAGTATCGACTCAACTATCAGAGGTAGTTGGCGTCATCGAGCGCCATCTCGAACCGACGTTGCTGGCCGTACATTTGTACGGCTCCGCAGTGGATGGCGGCCTGAAGCCACACAGCGATATTGATTTGCTGGTTACGGTGACCGTAAGGCTTGATGAAACAACGCGGCGAGCTTTGATCAACGACCTTTTGGAAACTTCGGCTTCCCCTGGAGAGAGCGAGATTCTCCGCGCTGTAGAAGTCACCATTGTTGTGCACGACGACATCATTCCGTGGCGTTATCCAGCTAAGCGCGAACTGCAATTTGGAGAATGGCAGCGCAATGACATTCTTGCAGGTATCTTCGAGCCAGCCACGATCGACATTGATCTGGCTATCTTGCTGACAAAAGCAAGAGAACATAGCGTTGCCTTGGTAGGTCCAGCGGCGGAGGAACTCTTTGATCCGGTTCCTGAACAGGATCTATTTGAGGCGCTAAATGAAACCTTAACGCTATGGAACTCGCCGCCCGACTGGGCTGGCGATGAGCGAAATGTAGTGCTTACGTTGTCCCGCATTTGGTACAGCGCAGTAACCGGCAAAATCGCGCCGAAGGATGTCGCTGCCGACTGGGCAATGGAGCGCCTGCCGGCCCAGTATCAGCCCGTCATACTTGAAGCTAGACAGGCTTATCTTGGACAAGAAGAAGATCGCTTGGCCTCGCGCGCAGATCAGTTGGAAGAATTTGTCCATTACGTAAAAGGCGAGATCACCAAGGTAGTCGGCAAATAA}
            \par}
                      &
    Streptomycin resistance gene for selection of constructs during Golden Braid assembly.
                      &
    BBa\_K4818060 \\
    \hline
\end{longtable}
